\documentclass[a4paper,11pt]{article}
\usepackage{amsfonts,amssymb,amsmath,amsthm,mathtools}
\usepackage{graphicx,xcolor,caption,subcaption,hyperref}
\usepackage[russian]{babel}
\usepackage[utf8x]{inputenc}

\usepackage{tikz}
\usetikzlibrary{calc,patterns}
\makeatletter
\newcommand\currentcoordinate{\the\tikz@lastxsaved,\the\tikz@lastysaved}
\makeatother

\RequirePackage[margin=2.05cm,top=2.5cm,bottom=1.7cm]{geometry}
\linespread{1.08} \parskip=0.1in \parindent=0.4in

\usepackage{fancyhdr} \pagestyle{fancy}
\renewcommand{\headrulewidth}{0.22mm}
\headheight=16pt \headsep=6mm \footskip=6mm

\def\defaultstyle{
	\fancyhead[LO,LE]{{\sffamily «Родные города» submission}}
	\fancyhead[RO,RE]{{\sffamily {\large Boris Zolotov,} {\footnotesize МКН СПбГУ}}}
	\fancyhead[CO,CE]{\ } \fancyfoot[CO,CE]{\thepage}
} \defaultstyle

\newtheorem{theorem}{Theorem}
\newtheorem{lemma}[theorem]{Lemma}
\newtheorem{stat}[theorem]{Statement}
\theoremstyle{definition}
\newtheorem{definition}{Definition}
\newtheorem{assu}{Assumption}

\begin{document}
\definecolor{tdefault}{RGB}{35,120,95}
\def\P{\mathcal P} \def\Ot{\tilde O}

\begin{center} \ \\ [1.5cm]
	{\Huge Boris Zolotov} \\ [0.35cm]
	{Recent results and ongoing research}
\end{center}

\section{All convex polygons that can be glued from regular pentagons}

What has already been done is the enumeration of all edge-to-edge gluings of regular pentagons satisfying the conditions of the Alexandrov's Uniqueness Theorem. In my work I solved the problem of establishing the graph structure of convex polyhedra that are glued from regular pentagons edge-to-edge.

I derived an upper bound for discrepancy in vertex coordinates between the unique convex polyhedron corresponding to a given polyhedral metric and a given approximate polyhedron:

\begin{theorem} \label{precision}
	Suppose $\mu$ is the maximum edge discrepancy between $P$ and $\P$, $\gamma$ is the maximum angle discrepancy between $P$ and $\P$, $\mathcal D$ is the maximum degree of a vertex of $P$. If $\mathcal D \gamma < \pi / 2$, then each vertex of $\P$ lies within an $r$--ball centered at the corresponding vertex of $P$, where
\begin{equation}
	r = E^2 \cdot  L \cdot 2 \sin ( \mathcal D \gamma / 2 ) + E \mu.
\end{equation} \end{theorem}

This implies a sufficient condition for the polyhedron to have a certain edge. I also developed the program that checks this condition.

Geometric methods can also be applied for establishing the graph structures that are non-simplicial. In particular I proved that two of the polyhedra glued from regular pentagons have quadrilateral faces. Applying the same type of argument, I gave an alternative proof for the existence of one of the edges in the polyhedron glued from 6 regular pentagons.

While the main outcome of this work is the full list of the polyhedra that are obtained by edge-to-edge gluings of regular pentagons, the methods for obtaining it are of independent interest and may be applied to other problems of the same flavour which gives this work a lot of implications related to considering different polygons and gluings composed from them.

The result has been presented at JCDCGGG$^3$ and is now awaiting publication in the special issue of Journal of Information Processing.

\section{Sublinear Explicit Incremental Planar Voronoi Diagrams}

We create a data structure that explicitly maintains the graph of a Voronoi diagram under insertion of new sites. The diagram is stored in adjacency list format on which primitive operations, including links and cuts, are performed.

It was observed previously that while there could be a linear number of changes to the embedded Voronoi diagram with each site insertion, this is not equivalent to the number of combinatorial changes to the graph structure of the Voronoi diagram.
It was then proved by showing that the amortized number of combinatorial changes is only $\Theta(N^{\frac 12})$ per site insertion.

This observation opened the possibility to maintain the Voronoi diagram graph in faster than linear time per insertion, which was consequently done for the restricted case where the sites are in convex position.

The new result is a data structure that explicitly maintains the graph of a Voronoi diagram of arbitrary point sites in the plane while allowing insertions of new sites in $\Ot (N^{\frac 34})$ expected amortized time, where $\Ot$ suppresses polylogarithmic terms. 

Previously, no sublinear method was known for this problem—this is what makes the result outstanding.

The result has been presented at JCDCGGG$^3$ and is now awaiting publication in the special issue of Journal of Information Processing.

\section{Ongoing research: \\
	\~Optimal-Time Incremental Combinatorial Voronoi Diagrams}

Continuing the previous work we asked ourselves the question: is it possible to store the graph of a Voronoi diagram and update it with each site insertion in time $\Ot(n)$ where $n$ is number of combinatorial changes per insertion? This would result in $\Ot(N^\frac12)$ expected amortised time per insertion, which is optimal possible time up to a polylogarithmic factor.

Turns out it must be possible to do using divide-and-conquer strategy. There is a known divide-and-conquaer algorithm for the static case of the Voronoi diagram, however it does not fit for semi-dynamic case and several significant changes have to be introduced.

My coauthors and I are now working on this and hoping to publish it or present it at a conference in the near future.

\end{document}