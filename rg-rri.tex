\documentclass[a4paper,11pt]{article}
\usepackage{amsfonts,amssymb,amsmath,amsthm,mathtools}
\usepackage{graphicx,xcolor,caption,subcaption,hyperref}
\usepackage[russian]{babel}
\usepackage[utf8x]{inputenc}

\addto\captionsrussian{\renewcommand\refname{References}}
\def\vsection#1{\vspace{-4mm}\section{#1}\vspace{-3mm}}

\usepackage{tikz}
\usetikzlibrary{calc,patterns}
\makeatletter
\newcommand\currentcoordinate{\the\tikz@lastxsaved,\the\tikz@lastysaved}
\makeatother

\RequirePackage[margin=1.65cm,top=2.15cm,bottom=1.4cm]{geometry}
\linespread{1.08} \parskip=0.09cm \parindent=0.28in

\usepackage{fancyhdr} \pagestyle{fancy}
\renewcommand{\headrulewidth}{0.22mm}
\headheight=16pt \headsep=6mm \footskip=6mm

\def\defaultstyle{
	\fancyhead[LO,LE]{{\sffamily «Родные города» submission}}
	\fancyhead[RO,RE]{{\sffamily {\large Boris Zolotov,} {\footnotesize МКН СПбГУ}}}
	\fancyhead[CO,CE]{\ } \fancyfoot[CO,CE]{\thepage}
} \defaultstyle

\newtheorem{theorem}{Theorem}
\newtheorem{lemma}[theorem]{Lemma}
\newtheorem{stat}[theorem]{Statement}
\theoremstyle{definition}
\newtheorem{definition}{Definition}
\newtheorem{assu}{Assumption}

\begin{document}
\definecolor{tdefault}{RGB}{35,120,95}
\def\P{\mathcal P} \def\Ot{\tilde O}

\begin{center}
	{\LARGE\bf Recent results and ongoing research}
\end{center} \vspace{-4.5mm}

\vsection{Sublinear Explicit Incremental Planar Voronoi Diagrams}

	Let $S \coloneqq \{s_1, s_2, \ldots, s_N\}$ be a set of $N$ distinct points in the plane. \emph{Voronoi diagram} of $S$ is the subdivision of the plane into $N$ \emph{cells}, one for each site in $S$, with the property that a point $q$ lies in the cell corresponding to a site $s_i$ if and only if $\mathrm{dist} (q, s_i) < \mathrm{dist} (q, s_j)$ for each $s_j \in S$ with $j \ne i$.

Voronoi diagrams have been of a great interest in TCS and computation geometry since their discovery~\cite{v-vd} until now~\cite{vd-new}. The aim of this work is to maintain the graph of a Voronoi diagram in time sublinear in $N$ under insertion of a new site.

We create a data structure that explicitly maintains the graph of a Voronoi diagram. The diagram is stored in adjacency list format on which primitive operations, including links and cuts, are performed.

It was observed previously that while there could be a linear number of changes to the embedded Voronoi diagram with each site insertion, this is not equivalent to the number of combinatorial changes to the graph structure of the Voronoi diagram.
It was then proved by showing that the amortized number of combinatorial changes is only $\Theta(N^{\frac 12})$ per site insertion.

This observation opened the possibility to maintain the Voronoi diagram graph in faster than linear time per insertion, which was consequently done for the restricted case where the sites are in convex position~\cite{incremental-vd}.

Our data structure allows insertions of new sites in $\Ot (N^{\frac 34})$ expected amortized time, where $\Ot$ suppresses polylogarithmic terms. Previously, no sublinear method was known for this problem—this is what makes the result outstanding. The result has been presented at JCDCGGG$^3$ and is now awaiting publication in the special issue of Journal of Information Processing.

\vsection{Ongoing research: \~Optimal-Time Incremental \\
	Combinatorial Voronoi Diagrams}

The time per insertion of a new site in our previous work is sublinear yet not optimal. We are now aiming at maintaining the Voronoi diagram while having optimal, $\Ot(N^{\frac12})$ time per insertion.

Our idea is to use divide-and-conquer strategy. There is a known divide-and-conquaer algorithm for the static case of the Voronoi diagram, but it does not fit directly for semi-dynamic case. My coauthors and I are now working on this and hoping to obtain the result soon.

\vsection{All convex polygons that can be glued from regular pentagons}

	A \emph{net} is a set of polygons equipped with a number of rules describing the way edges of these polygons must be glued to each other.

\vspace{-2mm} \begin{theorem}[Alexandrov, 1950,~\cite{alex}]
\label{thm:alexandrov}
	If a net is homeomorphic to a sphere and the sum of angles at each of its vertices is at most $360^\circ$ then there is a single convex polyhedron $P$ that can be glued from this net.
\end{theorem} \vspace{-3mm}

However, the proof of this theorem is highly non-constructive, and the algorithmic question of {\it construc-\linebreak ting} the polyhedron $P$, its 1-dimensional skeleton and coordinates of the vertices, remains open for around 60 years.

To get closer to the solution of this problem one can solve some of its restricted cases. One way is to consider a single polygon of a particular type, i.e. a regular polygon, and {\it edge-to-edge} gluings (an edge of a polygon $T_i$ needs to be glued to an entire other edge of another polygon $T_j$, possibly $j=i$) of its several copies~\cite{gfalop}.

What had already been done is the enumeration of all edge-to-edge gluings of regular pentagons satisfying the conditions of the Alexandrov's Uniqueness Theorem. In my work I solved the problem of establishing the graph structure of convex polyhedra that are glued from regular pentagons edge-to-edge.

I derived an upper bound for discrepancy in vertex coordinates between the unique convex polyhedron corresponding to a given polyhedral metric and a given approximate polyhedron, this implied a sufficient condition for the polyhedron to have a certain edge. I developed the program that checks this condition. I also applied geometric methods to establish the graph structures that are non-simplicial.

The main outcome of this work is the full list of the polyhedra that are obtained by edge-to-edge gluings of regular pentagons. However, the methods for obtaining it are of independent interest and may be applied to other problems of the same flavour which gives this work a lot of implications related to considering different polygons and gluings composed of them. The result has been presented at EGC 2019 and is now awaiting publication in the special issue of Journal of Information Processing.

\bibliographystyle{abbrv}
\bibliography{rg}

\end{document}